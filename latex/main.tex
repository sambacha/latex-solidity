\documentclass[runningheads]{llncs}
\usepackage{minted}


\usemintedstyle{vs}
\usepackage[utf8]{inputenc}
\usepackage[dvipsnames]{xcolor}
\definecolor{LightGray}{gray}{0.9}



\title{ \LaTeX{} Highlighting for Solidity and Yul}


\begin{document}

\author{Sam Bacha\inst{1}}
%
%\authorrunning{F. Author et al.}
% First names are abbreviated in the running head.
% If there are more than two authors, 'et al.' is used.
%
\institute{Manifold Finance, Inc \\
\email{\{sam\}@manifoldfinance.com} \\
\date{June 2022}}

\maketitle
\section{Usage}
\paragraph{Uses the python package, \em{pygments-lexer-solidity} package }

\section{Solidity}
\subsection{example.sol - Solidity}
\begin{minted}{lexer.py:SolidityLexer -x}




    // SPDX-License-Identifier: BSD-2-Clause

    pragma solidity ^0.6.0;
    pragma ABIEncoderV2;
    pragma experimental SMTChecker;
    /**********************************************************************
     *                             example.sol                            *
     **********************************************************************/
    
    // Code in this contract is not meant to work (or be a good example).
    // It is meant to demonstrate good syntax highlighting by the lexer,
    // even if otherwise hazardous.
    
    // Comments relevant to the lexer are single-line.
    /* Comments relevant to the code are multi-line. */
    
    library Assembly {
        function junk(address _addr) private returns (address _ret) {
            assembly {
                let tmp := 0
    
                // nested code block
                let mulmod_ := 0 { // evade collision with `mulmod`
                    let tmp:=sub(mulmod_,1) // `tmp` is not a label
                    mulmod_ := tmp
                }
                /* guess what mulmod_ is now... */
            _loop: // JIC, dots are invalid in labels
                let i := 0x10
            loop:
                // Escape sequences in comments are not parsed.
                /* Not sure what's going on here, but it sure is funky!
                 \o/ \o/ \o/ \o/ \o/ \o/ \o/ \o/ \o/ \o/ \o/ \o/ \o/ */
                mulmod(_addr, mulmod_, 160)
                
                0x1 i sub // instructional style
                i =: tmp /* tmp not used */
                
                jumpi(loop, not(iszero(i)))
                
                mstore(0x0, _addr)
                return(0x0, 160)
            }
        }
    }
    
    contract Strings {
        // `double` is not a keyword (yet)
        string double = "This\ is a string\nwith \"escapes\",\
    and it's multi-line. // no comment"; // comment ok // even nested :)
        string single = 'This\ is a string\nwith "escapes",\
    and it\'s multi-line. // no comment'; // same thing, single-quote
        string hexstr = hex'537472696e67732e73656e6428746869732e62616c616e6365293b';
    
        fallback() external payable virtual {}
    
        receive() external payable {
            revert();
        }
    }
    
    contract Types is Strings {
        using Assembly for Assembly;
    
        bytes stringsruntime = type(Strings).runtimeCode;
        
        // typesM (compiler chokes on invalid)
        int8 i8;           // valid
        //int10 i10;       // invalid
        uint256 ui256;     // valid
        //uint9001 ui9001; // invalid
        bytes1 b1;         //valid
        //bytes42 b42;     // invalid - M out of range for `bytes`
        
        // typesMxN (compiler chokes on invalid)
        fixed8x0 f8x0;           // valid
        fixed8x1 f8x1;           // valid
        fixed8x8 f8x8;           // valid
        //fixed0x8 f0x8;         // invalid since MxN scheme changed
        ufixed256x80 uf256x80;   // valid
        //ufixed42x217 uf42x217; // invalid - M must be multiple of 8, N <= 80
    
        // special cases (internally not types)
        string str; // dynamic array (not a value-type)
        bytes bs; // same as above
        //var v = 5; // `var` is a keyword, not a type, and compiler chokes
        uint unu$ed; // `var` is highlighted, though, and `$` is a valid char
    
        address a = "0x1"; // lexer parses as string
        struct AddressMap {
            address origin;
            address result;
            address sender;
            bool touched;
        }
        mapping (address => AddressMap) touchedMe;
    
        function failOnNegative(int8 _arg)
            private
            pure
            returns (uint256)
        {
            /* implicit type conversion from `int8` to `uint256` */
            return _arg;
        }
    
        // some arithmetic operators + built-in names
        function opportunisticSend(address k) private {
            /* `touchedMe[k].result` et al are addresses, so
               `send()` available */
            touchedMe[k].origin.send(uint256(k)**2 % 100 finney);
            touchedMe[k].result.send(1 wei);
            touchedMe[k].sender.send(mulmod(1 szabo, k, 42));
        }
    
        fallback() external payable override {
            /* inferred type: address */
            var k = msg.sender;
            /* inferred type: `ufixed0x256` */
            var v = 1/42;
            /* can't be `var` - location specifier requires explicit type */
            int negative = -1;
    
            // valid syntax, unexpected result - not our problem
            ui256 = failOnNegative(negative);
            
            // logic operators
            if ((!touchedMe[msg.sender].touched &&
                 !touchedMe[tx.origin].touched) ||
                ((~(msg.sender * v + a)) % 256 == 42)
            ) {
                address garbled = Assembly.junk(a + msg.sender);
    
                /* create a new AddressMap struct in storage */
                AddressMap storage tmp;
    
                // TODO: highlight all known internal keywords?
                tmp.origin = tx.origin;
                tmp.result = garbled;
                tmp.sender = msg.sender;
                tmp.touched = true;
    
                /* does this link-by-reference as expected?.. */
                touchedMe[msg.sender] = tmp;
                touchedMe[tx.origin] = tmp;
            }
            else {
                /* weak guard against re-entry */
                touchedMe[k].touched = false;
                
                opportunisticSend(k);
                
                delete touchedMe[k];
                /* these probably do nothing... */
                delete touchedMe[msg.sender];
                delete touchedMe[tx.origin];
            }
        }
    }
    
    /**
       \brief Examples of bad practices.
    
       TODO: This special NatSpec notation is not parsed yet.
    
       @author Noel Maersk
     */
    /// Triple-slash NatSpec should work.
    /// @title Some examples of bad practices.
    /// @author Noel Maersk
    /// @notice Very old, might've been "fixed" by obsoletion.
    /// @dev This is a dummy comment.
    /// @custom:unmaintained This code is not maintained.
    
    contract BadPractices {
        address constant creator; /* `internal` by default */
        address private owner; /* forbid inheritance */
        bool mutex;
    
        modifier critical {
            assert(!mutex);
            mutex = true;
            _;
            mutex = false;
        }
        
        constructor() external {
            creator = tx.origin;
            owner = msg.sender;
        }
    
        /* Dangerous - function public, and doesn't check who's calling. */
        function withdraw(uint _amount)
            public
            critical
            returns (bool)
        { /* `mutex` set via modifier */
            /* Throwing on failed call may be dangerous. Consider
               returning false instead?.. */
            require(msg.sender.call.value(_amount)());
            return true;
        } /* `mutex` reset via modifier */
        
        /* fallback */
        fallback() external payable {
            /* `i` will be `uint8`, so this is an endless loop
               that will consume all gas and eventually throw.
             */
            for (var i = 0; i < 257; i++) {
                owner++;
            }
        }
    
        /* receive()?.. nah, why bother */
    }
    
    /* /* /* /* /* /* /* /* /* /* /* /* /* /* /* /* /* /* /* /* /* /* /*
    // A regular multi-line comment closure, including an escaped variant as
    // demonstrated shortly, should close the comment; note that the lexer
    // should not be nesting multi-line comments.
    //
    // If the comment is still shown as "open", then a
    //
    //                !!!!!!!!!!!!!!!!!!!!!!!!!!!!!!
    //                !!! MALICIOUS CODE SEGMENT !!!
    //                !!!!!!!!!!!!!!!!!!!!!!!!!!!!!!
    //
    // can be erroneously thought of as inactive, and left unread.
    // In fact, the compiler will produce executable code it, possibly
    // overriding the program above.
    //
    // It is imperative that syntax highlighters do parse it if either of
    // `* /` or `\* /` (with space removed) are present.
    //
    // Now, let's party! :) \*/
    
    contract MoreBadPractices is BadPractices {
        uint balance;
    
        fallback() external payable override {
            balance += msg.value;
            if (!msg.sender.send(this.balance / 10)) throw;
            balance -= this.balance;
        }
    }
    
    /*
    // Open comment to EOF. Compiler chokes on this, but it's useful for
    // highlighting to show that there's an unmatched multi-line comment
    // open.
    
    contract CommentToEndOfFile is MoreBadPractices {
        fallback() external payable override {}
    }
    
\end{minted}
\newpage
\subsection{example.yul - Yul}
\begin{minted}[
frame=lines,
framesep=2mm,
baselinestretch=1.2,
bgcolor=LightGray,
fontsize=\footnotesize,
linenos
]{lexer.py:YulLexer -x}
{
    // my function
    function power(base, exponent) -> result
    {
        switch exponent
        case 0 { result := 1 }
        case 1 { result := base }
        default
        {
            result := power(mul(base, base), div(exponent, 2))
            switch mod(exponent, 2)
                case 1 { result := mul(base, result) }
        }
    }
}
\end{minted}
\footnote{Taken from example.yul}

\newpage

\colorlet{LightRubineRed}{RubineRed!70}
\colorlet{Mycolor1}{green!10!orange}
\definecolor{Mycolor2}{HTML}{00F9DE}

\pagecolor{black}
\color{white}% set the default colour to white

This document presents several examples showing how to use the \texttt{pygments-lexar-solidity} and \texttt{xcolor} package 
to provide syntax highlighting for the programming languages Solidity and Yul as well as changing the colour of \LaTeX{} page elements.

\begin{itemize}
\item \textcolor{Mycolor1}{First item}
\item \textcolor{Mycolor2}{Second item}
\end{itemize}

\noindent
{\color{LightRubineRed} \rule{\linewidth}{1mm}}

\noindent
{\color{RubineRed} \rule{\linewidth}{1mm}}
\end{document}
